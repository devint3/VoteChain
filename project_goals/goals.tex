\documentclass[titlepage]{article}

\title{\Huge{\textbf{
SFWR 4G06 Capstone\\VoteChain
}}\vspace{8cm}}
\author{\textit{Jadees Anton} - <email> - <studentnum>\\
\textit{Connor Hallett} - <email> - <studentnum>\\
\textit{Nick Lago} - <email> - <studentnum>\\
\textit{Spencer Lee} - <email> - <studentnum>\\
\textit{Connor Sheehan} - sheehacg@mcmaster.ca - 1330964\\
\textit{Devin Thomas} - thomad4@mcmaster.ca - 1146745\\
}
\date{Friday, October 14\textsuperscript{th}, 2016}

\begin{document}
\maketitle

\section{Problem Statement}
The past 20 years have seen a rapid advancement in cryptography, computer security and web technologies. Despite these advancements, the area of online voting and elections have not seen a secure, auditable application developed. \textit{BitCoin}, the digital cryptocurrency, has managed to provide an auditable, secure public ledger of every transaction in the history of the currency. This technology, known as the \textit{block chain distributed database system} could be applied to any public and immutable record, including elections.

\section{Importance of Work}
A typical Canadian citizen can manage most of their daily lives via web services and software applications, including banking/financial planning, health records and personal correspondances. Moving voting to an online medium would provide all of the benefits of these applications such as convenience, decreased wait times and more. Using a publically available ledger of votes would also reduce the risk and impact of voter fraud, and allow citizens in jurisdictions with a history of corruption and rigged elections to cast votes with confidence in their merit.

\section{How it Works}
1	Election set up committee
  a.	Goes to portal, creates an account for the coming election
  b.	Enters the parameters for the election
    i.	Voting timetable
    ii.	Voters
    iii.	Candidates
    iv.	Notifications
  c.	Confirms information
2	Voter
  a.	Receives e-mail for election
  b.	Receives notifications if applicable 
  c.	Enters portal using credentials given in e-mail/or physically (?)
  d.	Votes for candidate
  e.	Receives a confirmation
  f.	Receives a notification when election is over with results


\section{Scope of Work and Term Goals}
\begin{enumerate}
\item Implement a BlockChain public ledger for recording votes.
\item Implement a front-end portal to cast votes and see election results in real-time.
\item Implement a vote tallying software for first-past-the-post style voting.
\item Validate and verify the security and auditability of the application.
\end{enumerate}

\section{Future Goals}
\begin{enumerate}
\item Implement different vote tallying algorithms to meet different user and cultural needs
\item Implement a full standalone front end voting booth system
\end{enumerate}

\end{document}
